\documentclass{sbposter}
\usepackage{graphicx}
\usepackage[normalem]{ulem}
\usepackage[inline]{enumitem}
\setlist{noitemsep} 
\usepackage{booktabs} % tables
\usepackage{caption}

% Debug
% \font\nullfont=cmr10

% tikzposter title information
% For long title:
\title{\parbox{0.6\linewidth}{Title of your paper - it can be long and break over several lines}}
% For short title:
%\title{Title of your paper}
\author{First Author, Second Author and Possibly More}
\institute{Språkbanken Text, University of Gothenburg}

\begin{document}
\maketitle

% Introduction - full page width
\block{Abstract}{
    Abstract or introduction spanning the whole page, that is, not placed in the two columns.
}

% Two column format
\begin{columns}
\column{0.5}

% Block - normal
\block{A block}{

    Each block can contain:

    \begin{itemize}
        \item text, with \textit{italic} and \textbf{bold} text
        \item lists, like this one
        \item tables (but put your \texttt{tabular}s inside \texttt{minipage}s, not \texttt{table}s!)
        \item images, like the one in the block below
    \end{itemize}
}

% Block - with an image
\block{Another block with an image}{
    \begin{center}
        \includegraphics[width=0.25\textwidth]{images/sprakbanken_text_gu_en.pdf}
%        \includegraphics[width=0.25\textwidth, natwidth=14cm,natheight=23.4cm]{images/sprakbanken_text_gu_en.svg}
    \end{center}
}

% Block - with table
\block{A third 1-column block with a tiny table}{
    \begin{minipage}{\linewidth}
        \centering
        \resizebox{0.3\columnwidth}{!}{
            \begin{tabular}{@{}ll@{}}
                \toprule
                Column 1 & Column 2 \\ \midrule
                Cell 1A  & Cell 2A  \\
                Cell 1B  & Cell 2B  \\
            \end{tabular}
        }
        \captionof{table}{This is a caption}
    \end{minipage}
}

% Next column
\column{0.5}

% Block - with sub blocks (consisting of a title and indented text)
\block{A block with sub-blocks}{
    Each block can contain however many inner blocks.

    \innerblock{An inner block}{
        This is the first inner block.
    }

    \innerblock{Another inner block}{
        This is the second and last inner block.
    }
}

% Block - without a title
\block{}{
    Blocks don't necessarily always have a title.
}

% Block - with an orange frame
\begingroup
\colorlet{framecolor}{sborange}
\block{A block with an orange frame}{
    This sticks out a little bit more.
}
\endgroup

\end{columns}
\end{document}

